\chapter{Konzept}

Um die in den Grundlagen beschriebenen Sicherheitsstandards, Protokolle und Integrationslösungen auf ihre Standhaftigkeit in Bezug auf die IT-Schutzziele zu analysieren, werden die Protokolle und Systeme in ihrem Aufbau untersucht und mögliche Schwachstellen herausgearbeitet, daraus hervorgehende Risiken beschrieben und erforderliche Maßnahmen empfohlen. Die \ac{RAMI4.0} beschreibt ein Referenzmodell für Industrie 4.0 Umgebungen. Bereits etablierte Lösungen bestehen aus heterogenen, individuellen Netzwerklandschaften. Um eine Untersuchung der vorhandenen Systeme im neuen Umfeld durchzuführen, müssen verschiedene Faktoren, wie Infrastruktur oder besondere Anforderungen an die Systeme mit einbezogen werden. Das folgende Kapitel dient der Beschreibung der Vorgehensweise bei der Analyse der Netzwerkkommunikation und deren Komponenten.

\section{Anforderungen}
Die Kommunikation in Industrie 4.0 Umgebungen findet über diverse Übertragungsmedien, Protokolle und Software statt. Um die beteiligten Komponenten auf Schwachstellen zu untersuchen, werden zuerst die Anforderungen für eine sichere Kommunikation herausgearbeitet, um eine Grundlage der Analyse zu schaffen.

\subsection{Grundlegende Anforderungen}
Aufgrund der sehr unterschiedlichen Einsatzbereiche von Industrie 4.0 Systemen, unterscheiden sich auch dementsprechend deren Anforderungen. Aus den Referenzmodellen \ac{RAMI4.0} und \ac{IIRA} lassen sich grundsätzlich drei grundlegende Anforderungen an den Übertragungskanal ableiten \cite{BMWiNeCon2016}.

\subsubsection{Sicherheit}
TODO - Hierunter fallen die Bereiche a) Netzsicherheit und Datensicherheit, b) Sichere Identitäten und c) funktionale Sicherheit. Die Punkte a) und b) werden in der AG3 der Plattform Industrie 4.0 adressiert [6], [7]. Die UAG Netzkommunikation arbeitet bzgl. dieser Punkte mit der AG3 zusammen. Zum Thema „Security und funktionale Sicherheit“ arbeitet die AG3 mit dem DKE-TBINK AK IT Security und Security by Design zusammen. Hinsichtlich funktionaler Sicherheit gibt es Anforderungen von Seiten IEC 61784-3. Diese müssen bei der Definition neuer Systeme berücksichtigt werden.

\begin{itemize}
    \item Netzsicherheit und Datensicherheit
    \item Sichere Identitäten
    \item funktionale Sicherheit
\end{itemize}

\subsubsection{Verfügbarkeit}
Die ständige Verfügbarkeit von Daten und Diensten spielst in der Industrie 4.0 eine bedeutende Rolle, um den Datenaustausch zwischen zwei Kommunikationspartnern im Netz jederzeit zu ermöglichen. Als Verfügbarkeit wird die Wahrscheinlichkeit bezeichnet, dass ein System innerhalb eines bestimmten Zeitraumes erreichbar ist. Ein System gilt als verfügbar, wenn es erreichbar ist und die für es vorgesehenen Aufgaben erledigen kann.

Die Verfügbarkeit eines Systems wird in Verfügbarkeitsklassen gegliedert. Diese beschreiben Verfügbarkeitswahrscheinlichkeiten von 99\% ( Verfügbarkeitsklasse 2 ) bis 99,9999\% ( Verfügbarkeitsklasse 6 ). Eine exakte Definition, wann ein System hochverfügbar ist, gibt es nicht - TODO ref. Im Allgemeinen wird ab Verfügbarkeitsklasse 3 ( 99,99\% ) von Hochverfügbarkeit gesprochen. Industrie 4.0 Systeme sind meist Hochverfügbar.

\subsubsection{\ac{QoS}}
TODO - Es ist die originäre Aufgabe der Datenkommunikation, Distanzen zu überwinden - egal wie weit die Kommunikationspartner voneinander entfernt sind: Effizienz- und produktivitätssteigernd ist sowohl die Überwindung von wenigen Zentimetern per Near Field Communication (NFC) als auch die Datenübertragung rund um den Globus durch verschiedene Netze; nicht zu vergessen: Teleservices zur Unterstützung bei Inbetriebnahmen, zum Remote Debugging und zum Fernwirken. Die Qualitätsanforderungen an Kommunikationsnetze (wired und wireless) sind: hochverfügbare, homogene Netze; garantierte Bandbreiten für die sehr unterschiedlichen Anwendungen (Bild 1); verbindliche Dienstgüte (Quality of Service, QoS) [1]; standardisierte Dienste (z.B. mobilfunkproviderübergreifende SMS-Bestätigung). 

TODO - ref. \cite{torscht2014} und IEEE 802.1p

Industrie 4.0 Dienste basieren auf IP-Netzen. Sie bilden nach dem OSI-Modell eine höhere Schicht im Netz. Somit setzt sich die Güte eines Dienstes aus der Übertragungsgüte der unteren Schichten des  OSI-Modells sowie der \ac{QoS}-Parameter des Network Layer ( IP-Ebene ) zusammen. In IP-Netzen wird der Einfluss auf die \ac{QoS} in folgenden Parametern beschrieben:
\begin{itemize}
    \item Latenzzeit: Dauer der Paketübertragung
    \item Jitter: Abweichung der Latenzzeit von ihrem Mittelwert
    \item Paketverlustrate: Wahrscheinlichkeit des Verlusts von IP-Paketen während der Übertragung
    \item Durchsatz: gemittelte Datenmenge pro Zeiteinheit
\end{itemize}

\subsection{Besondere Anforderungen}
\subsubsection{zeitkritische Prozesse}
\subsubsection{Migration vorhandener Systeme}

\section{Komponenten}
Die beschriebenen Anforderungen müssen, um eine sichere Netzwerkkommunikation zu gewährleisten, von allen beteiligen Komponenten der Umgebung integriert und umgesetzt werden. Industrie 4.0 Umgebungen können in unterschiedlichster Form ausgeprägt sein. Die Umsetzung der Hard- und Softwarekomponenten hängt von den zu übertragenden Daten, dem Übertragungsmedium, der Übertragungsdistanz und vorausgesetzten Dienstgüte ab. Somit werden die zu analysierenden Komponenten in Hard- und Softwarekomponenten gegliedert.

\subsection{Hardware}
\subsubsection{Übertragungskanal}
\subsubsection{Infrastruktur}

\subsection{Software}
\subsubsection{Netzwerkstack}
\subsubsection{\ac{IoT}-Protokolle}
\subsubsection{\ac{IIoT}-Protokolle}

\section{Umsetzung}
TODO - testen im Testsystem