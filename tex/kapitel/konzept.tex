\chapter{Konzept}

Um die in den Grundlagen beschriebenen Sicherheitsstandards, Protokolle und Integrationslösungen auf ihre Standhaftigkeit in Bezug auf die IT-Schutzziele zu analysieren, werden die Protokolle und Systeme in ihrem Aufbau untersucht und mögliche Schwachstellen herausgearbeitet, daraus hervorgehende Risiken beschrieben und erforderliche Maßnahmen empfohlen. Die \ac{RAMI4.0} beschreibt ein Referenzmodell für Industrie 4.0 Umgebungen. Bereits etablierte Lösungen bestehen aus heterogenen, individuellen Netzwerklandschaften. Um eine Untersuchung der vorhandenen Systeme im neuen Umfeld durchzuführen, müssen verschiedene Faktoren, wie Infrastruktur oder besondere Anforderungen an die Systeme mit einbezogen werden.

Im ersten Schritt sollen die verschiedenen genutzten Übertragungsmedien und deren Protokolle untersucht werden. - TODO

\section{Infrastruktur/Integrationslösungen}

\section{Übertragungskanäle}
Kabel - Funk - LAN - WAN
\begin{itemize}
    \item Instabile Übertragungskanäle
    \item \ac{LPWA}
\end{itemize}

\section{Grundlegende Anforderungen}
\subsection{Sicherheit}
\subsection{Verfügbarkeit}
\subsection{QoS}

\section{Besondere Anforderungen}
\subsubsection{zeitkritische Prozesse}
\subsubsection{TODO}
