\chapter{Anwendungsszenarien}
\label{Anwendungsszenarien}
Im folgenden Kapitel wird die Netzwerksicherheit verschiedener Bestandteile des \ac{TCP}/\ac{IP} Referenzmodells anhand der in \autoref{Analyse} durchgeführten Analyse der Netzwerkkommunikation überprüft. Dies wird mit Hilfe des Testsystems (\cite{Weber2018}) ermöglicht und soll die Auswirkungen von Bedrohungen der genutzten Technologien darstellen. Die beschriebenen Anwendungsszenarien wurden gewählt, um die Nutzung eines Schichtenmodells im Netzwerk hervorzuheben. Die Einführung eines neuen Protokolls auf einer Ebene des Referenzmodells hat keine Auswirkungen auf die Funktionsweise der anderen, genutzten Schichten. Jede Schicht im Netzwerk kann durch Sabotage die Manipulation des Netzwerk ermöglichen.

Die Umsetzung der Anwendungsszenarien SYN-Flood, \ac{ARP} Spoofing und \ac{DNS} Amplifictation war nicht möglich, da die Manipulation des Netzwerks und dessen Kommunikation im gegebenen Testsystem aufgrund der genutzten Software nur begrenzt ausführbar war und somit die Bereitstellung einer Netzwerkinfrastruktur wesentlicher Bestandteil der Thesis darstellte. Dies wird in \autoref{Konzept:Anpassungen} näher erläutert.

Nach Abwägung der Faktoren zeitlicher Aufwand, Umsetzbarkeit und Darstellungsmöglichkeit und Mehrwert für das System, wurde sich für die folgenden Anwendungsszenarien entschieden.

\section{\ac{OPC UA} Kommunikation}
\label{Anwendungsszenarien:OPC UA Kommunikation}
Die Kommunikation des \ac{OPC UA} Protokolls findet laut Spezifikation, wie in \autoref{Analyse:OPC UA} beschrieben, immer im \textit{Secure Channel} statt. Der \textit{Secure Channel} stellt verschiedene \textit{Security Policies} für unterschiedliche Anwendungsfälle zur Verfügung. Administratoren müssen abwägen in wie Fern Ressourcen und Latenz für die Absicherung der Kommunikation im Netzwerk gewährleistet werden können. Die Kommunikation zwischen \ac{OPC UA} Komponenten und die verschiedenen Sicherheitsprofile können im vorhandenen Testsystem (\cite{Weber2018}) untersucht werden. Hierfür wird das Netzwerkanalysetool Wireshark\footnote{Link: https://www.wireshark.org/} genutzt, um den Netzwerkverkehr zwischen den Komponenten abzuhören. In Verbindung mit dem \ac{OPC UA} \textit{Secure Channel} wird das System erweitert, um verschiedene Sicherheitsprofile für die Kommunikation bereitzustellen und somit die Auswirkungen einer Fehlkonfiguration darstellen zu können.

\section{\ac{MitM}}
\label{Anwendungsszenarien:MitM}
Im vorhandenen System soll die Darstellung eines \ac{MitM} Angriffs ermöglicht werden. Dieser wird mit Hilfe eines Rogue \ac{DHCP} Servers durchgeführt. Es soll ermöglicht werden, die Netzwerkkonfiguration einer Komponente so zu Manipulieren, um die Kommunikation mithören zu können. Dies beinhaltet die Bereitstellung einer Netzwerkinfrastruktur inklusive \ac{DHCP} und \ac{DNS}.

\section{Manipulation von ungesichertem Netzwerkverkehr}
\label{Anwendungsszenarien:Manipulation von ungesichertem Netzwerkverkehr}
Im dritten Anwendungsszenario sollen die durch den durchgeführten \ac{MitM} Angriff gewonnenen Informationen genutzt werden, um die Funktionalität eines weiteren Systems im Netzwerk zu stören und somit direkt Einfluss auf einen Prozess in einem Industrienetzwerk nehmen. Aufgrund der in \autoref{Grundlagen:IoT/IIoT} beschriebenen Faktoren und der immer weiteren Vernetzung ressourcenschwacher Komponenten soll das Anwendungsszenario anhand einer minimalen \ac{IIoT} Komponente am Protokoll \ac{CoAP} durchgeführt werden.