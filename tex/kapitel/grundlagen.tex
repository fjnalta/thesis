\chapter{Grundlagen}

\section{Historie}
Industrie 3.0 -> Industrie 4.0 - Kommunikation über Unternehmensgrenzen, Kommunikation nicht mehr über ERP und MES sondern direkt von unteren Schichten, wie z.B. Maschinen oder Komponenten

\section{Kommunikation in Industrie 4.0}
Im Gegensatz zur I3.0, in welcher Daten auf lokaler Ebene oder zwischen einzelnen internen Unternehmensebenen ausgetauscht wurden, stellt in der I4.0 der Austausch von Daten und Informationen über Unternehmensgrenzen hinweg eine wesentliche Herausforderung dar. Dabei findet die Kommunikation nicht mehr über ein Enterprise-Resource-Planning-System (ERP) statt, sondern auch direkt von einer darunterliegenden Ebene, wie z. B. einer Maschine mit ihrem Lieferanten. Durch diese enge Vernetzung können sowohl Menschen, als auch Maschinen die Kommunikationspartner sein.

\subsection{Anforderungen}
\subsection{Komponenten einer I4.0 Architektur}
\subsection{Kommunikationsstrukturen}
\subsubsection{End2End}
\subsubsection{Gateways}
\subsubsection{Publish-Subscribe}
\subsubsection{Kommunikation mit Netzwerk als Partner}

\subsection{Schutzziele}
Für diese neuen Szenarien gelten weiterhin die klassischen Schutzziele der Vertraulichkeit, Integrität und Verfügbarkeit. 

Des Weiteren werden Bestellungen oder Logistikprozesse durch I4.0 Kommunikation abgewickelt. Diese stellen einen rechtlichen Rahmen dar, welcher weitere Schutzziele beinhaltet:
\begin{itemize}
    \item Authentizität
    \item Nichtabstreitbarkeit
    \item Verbindlichkeit
    \item Zurechenbarkeit
  \end{itemize}

TODO – Vertraulichkeit, Integrität, Verfügbarkeit, Authentizität, Nichtabstreitbarkeit, Verbindlichkeit, Zurechenbarkeit
