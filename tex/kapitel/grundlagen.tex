\chapter{Grundlagen}

\section{Historie}
Seit dem Beginn des Industriezeitalters um 1800, welches mit der Mechanisierung (Industrie 1.0) startete, befindet sich die Industrie in einem stetigen Wandel. Sie entwickelte sich um 1900 durch die Massenproduktion zur Industrie 2.0 und in den 1970er Jahren durch die Automatisierung zur Industrie 3.0. Die Einteilung der Industriezeitalter ist durch tiefgreifende Veränderungen im technologischen Fortschritt möglich, welche auch als industrielle Revolution bezeichnet werden. Aktuell befinden wir uns in der Phase der 4. industriellen Revolution.

\subsection{1. industrielle Revolution}
Die 1. industrielle Revolution fand mit der Erfindung der Dampfmaschine statt. Sie ermöglichte es Eisenbahnen und Dampfschiffe sowie verschiedene Maschinen im Kohleabbau oder in Textilfabriken anzutreiben und trug massiv zur Industrialisierung und der Entstehung der Industrie 1.0 bei. Nach und nach wurden immer mehr Produktionsanlagen errichtet und somit Arbeitsplätze in Infrastruktur, Textilfabriken, Häuserbau, Kohleabbau und anderen Bereichen geschaffen.

\subsection{2. industrielle Revolution}
Die Erforschung der Elektrizität im 19. Jahrhundert war der Auslöser der 2. industriellen Revolution. Nachdem ab 1830 die Gesetze der Elektrotechnik bekannt waren, fand die Elektrizität eine breite Anwendung in der Industrie und im Alltag. Im Jahr 1913 führte Henry Ford das Fließband in der Automobilbranche ein. Im Zuge dessen musste jeder Arbeiter nur noch einen Arbeitsschritt erledigen, welches einerseits die Produktion wesentlich beschleunigte und eine Massenproduktion ermöglichte und andererseits eine hohe Spezialisierung der einzelnen Arbeitskräfte für ihre bestimmte Aufgabe erforderte.

Außerdem wurde es durch die Luftfahrt möglich Produkte wie Autos, Kleidung und Lebensmittel über Kontinente hinweg immer schneller zu transportieren und zu handeln.

\subsection{3. industrielle Revolution}
Die 3. industrielle Revolution fand in den 1970er Jahren statt. Sie ist durch eine sukzessive (Teil-) Automatisierung der Prozesse und durch den Einzug der IT in die Industrie- und Verbraucherwelt geprägt. In den 1940er Jahren wurden die ersten Rechenmaschinen und programmierbare Steuerungen in Unternehmen eingesetzt. In den 1970er Jahren zog der Computer auch in den Privatbereich ein, wurde zunehmend beliebter und schaffte einen neuen Industriezweig. Der Fertigungsprozess in Fabriken wurde mehr und mehr von Maschinen übernommen.

Durch den zunehmenden Einsatz von IT in Unternehmen entstand immer mehr Kommunikation zwischen Menschen und Maschinenn. Diese Kommunikation und die anfallenden Daten wurden jedoch nur unternehmensintern verarbeitet. Es gab nur wenige Schnittstellen nach außen.

\begin{figure}[h]
  \centering
  \includegraphics[width=15cm]{kommunikationsbeziehungen-i30}
  \caption{Kommunikationsbeziehungen in einer Industrie 3.0 Umgebung - TODO ref. sichere unternehmensübergreifende Kommunikation}
  \label{Kap2:Industrie3.0-Kommunikation}
\end{figure}

\clearpage

\subsection{4. industrielle Revolution}
Das Ende des 20. Jahrhunderts gilt als der Beginn der 4. industriellen Revolution. Das Kennzeichen dieser Phase ist die zunehmende Digitalisierung. Mit ihr geht die technische Vernetzung physischer Gegenstände, dem \ac{IoT}, einher. Mehr und mehr Geräte oder Gegenstände besitzen die Möglichkeit aktiv durch Datenaustausch oder passiv z. B. mit Hilfe eines Bar- oder QR-Codes mit der digitalen Welt zu kommunizieren und somit eine fortschreitende Automatisierung sowie Individualisierung zu ermöglichen. 

Im Gegensatz zur Industrie 3.0 sollen Maschinen autonom, auch über Unternehmensgrenzen hinweg, miteinander kommunizieren können um gesamte Geschäftsprozesse zu übernehmen. Dies setzt eine Öffnung der Unternehmen nach außen voraus.

\begin{figure}[h]
  \centering
  \includegraphics[width=15cm]{kommunikationsbeziehungen-i40}
  \caption{Kommunikationsbeziehungen in einer Industrie 4.0 Umgebung - TODO ref. sichere Unternehmensübergreifende Kommunikation}
  \label{Kap2:Industrie4.0-Kommunikation}
\end{figure}

\clearpage

Diese Entwicklung erzeugt durch die ständige Kommunikation eine ernome Menge an Daten, welche den Anforderungen der IT-Sicherheit gerecht werden müssen, um Verbraucher und Unternehmen zu schützen.

\section{aktueller Stand der Technik}
Der Prozess der vierten industriellen Revolution ist ein stetiger, nicht abgeschlossener Prozess. Aktuell werden die ersten Smart Factories der Industrie errichtet und erste smarte Einkaufsmöglichkeiten, wie Amazon Go und TODO - siehe Trumpf, für den Endverbraucher geschaffen. Diese Fabriken und Filialen stellen die ersten ihrer Art dar und dienen als Prototypen. Das Ziel des Wandels in der Strukturierung und Organisation der Produktion in Unternehmen ist eine immer weitere Automatisierung der Prozessabwicklung bis hin zu autonom arbeitenden Fabriken. Für kritische Infrastrukturen, wie z. B. im Energie-, Wasser-, Transport- und Gesundheitssektor existiert diese Verbindung bereits.

Die Umsetzung dieser Innovationen basiert hauptsächlich auf dem Fortschritt der \ac{IT} und dem Einzug der Internet-Technologien in die Industrie. Diese Entwicklung macht es möglich immer schneller Informationen auszutauschen, größere Datenmengen zu analysieren und diese zu verarbeiten. In der Industrie entstehen dadurch u. a. die folgenden Chancen:

\begin{itemize}
  \item Die Kommunikationsinfrastruktur wird in Zukunft in Produktionssystemen so preiswert sein, dass sie sinnvoll für Konfiguration, Service, Diagnose, Bedienung und Wartung genutzt werden kann.
  \item Die Produktionssysteme werden mehr und mehr mit einem Netz verbunden, erhalten dort eine digitale Identität, werden somit such- und analysierbar und besitzen die Möglichkeit Daten über sich selbst zu veröffentlichen. 
  \item Maschinen und Anlagen speichern ihre Zustände in ihrer digitalen Identität im Netz. Diese Zustände sind aktuell, aktualisierbar und zunehmend vollständig. Sind im Netzwerk viele solcher Identitäten vorhanden, können die Daten effizient abgerufen und ausgetauscht werden.
  \item Softwaredienste werden über das Netz verknüpft werden und können somit automatisiert individuelle Aufgaben durch die direkte Kommunikation der Systeme erledigen. Eine solche individuelle Wertschöpfung war bisher nur unwirtschaftlich oder gar nicht möglich.
\end{itemize}

Diese Veränderungen im Wertschöpfungsprozess und die ständige Kommunikation der Systeme bereiten jedoch auch Probleme. Es entstehen große Mengen an Daten, welche u. a. über einen unsicheren Kanal verbreitet werden sollen. Des weiteren sind viele vorhandene Produktionsanlagen nicht für diese Form von vermaschter Kommunikation entwickelt worden. Diesen Problemen wird aktuell durch die Entwicklung von Industriestandards und \ac{M2M}-Protokollen, wie z. B. die \ac{OPC UA} entgegengewirkt. Um vorhandene Anlagen weiterhin nutzen zu können, werden Gateways genutzt. (TODO Trumpf ref.)

\section{Industrie 4.0}
Der Begriff Industrie 4.0 wurde erstmals auf der Hannover Messe 2011 verwendet (\cite{drath2014}) und soll das Ergebnis der 4. industriellen Revolution darstellen. Der Grundgedanke hinter Industrie 4.0 ist die flächendeckende Vernetzung von Informations- und Kommunikationstechnik zu einem Internet der Dinge, Dienste und Daten (\cite{Spath2013}). Diese Vernetzung soll einen ständigen Informationsaustausch zwischen den Komponenten ermöglichen. Jede Komponente des \ac{IoT} soll als \ac{CPS} arbeiten. Ein \ac{CPS} besitzt neben seiner realen Identität eine digitale Identität, über welche es ständig mit anderen \ac{IoT}-Geräten kommunizieren kann. Kunden- und Maschinendaten werden miteinander vernetzt \cite{rami2016}.

\begin{figure}[h]
  \centering
  \includegraphics[width=15cm]{internet-der-dinge}
  \caption{Das Internet der Dinge - \cite{rami2016}}
  \label{Kap2:Das Internet der Dinge}
\end{figure}

\clearpage

Für Unternehmen bedeutet dies einen Wechsel von einer linearen Prozesskette hin zu einem vermaschten Netzwerk, in dem jede Komponente mit dem gesamten Netzwerk kommunizieren kann. Dies beinhaltet die Vernetzung der Komponenten auf horizontaler und vertikaler Ebene. Die vertikale Ebene stellt die technischen Komponenten dar und wird durch die Automatisierungspyramide beschrieben. Die horizontale Ebene beschreibt die wirtschaftlichen Geschäfts- bzw. Produktionsprozesse und besteht u. a. aus: Einkauf, Lieferanten, Produktionsplanung, Logistik, Sequenzierung und Lagerverwaltung. Das Ziel ist die Vernetzung aller Beteiligten.

\begin{figure}[h]
  \centering
  \includegraphics[width=15cm]{horizontaleVertikaleIntegration}
  \caption{horizontale und vertikale Integration - TODO ref. HP Industry-of-things siehe bookmark}
  \label{Kap2:horizontale und vertikale Integration}
\end{figure}

\clearpage

\subsection{\ac{IoT}/\ac{IIoT}}
\ac{IoT} beschreibt ein verbraucherorientiertes Konzept für die Nutzung von digitalisierten und vernetzten Systemen. Hierbei werden die physischen Systeme virtuell abgebildet. Dies wird genutzt, um die Effektivität der Systeme zu verbessern und intelligente Services zu nutzen. Das \ac{IIoT} beschreibt den Gebrauch von \ac{IoT}-Technologien im industriellen Raum.

Das \ac{IoT} ist ein wesentlicher Bestandteil der Industrie 4.0, welche Netzwerke aus Systemen, Daten und Dienstleistungen herstellt, in denen diese Komponenten miteinander kommunizieren. Für die Kommunikation haben sich, je nach Anforderungen, verschiedene Protokolle, wie z.B. \ac{HTTP}, \ac{CoAP}, \ac{XMPP} und \ac{MQTT}, etabliert. Jedes dieser Protokolle besitzt für spezifische Anforderungen wie Skalierbarkeit, vorhandene Ressourcen, Echtzeitkommunikation oder Sicherheit Vor- und Nachteile. 

\subsection{Automatisierungspyramide}
Die Automatisierungspyramide stellt die beteiligten Systeme und Softwarekomponenten eines automatisierten Prozesses systematisch dar. Diese beginnen, ausgehend vom Kundenauftrag und der betriebswirtschaftlichen Planung der Produktion auf der Unternehmensebene im \ac{ERP} System. Die Ergebnisse der Planung werden an das \ac{MES} übergeben, welches die verschiedenen Fertigungs- oder Logistikaufträge generiert. Die Aufträge werden anschließend auf der Prozessleit- (\ac{SCADA}), Steuerungs- (\ac{SPS}) und Feldebene (Ein-/Ausgangssignale) bearbeitet.

\begin{figure}[h]
  \centering
  \includegraphics[width=10cm]{automatisierungspyramide}
  \caption{Automatisierungspyramide - TODO ref. Langmann,2004}
  \label{Kap2:Automatisierungspyramide}
\end{figure}

\clearpage

Während die oberen Schichten der Pyramide (\ac{ERP} und \ac{MES}) durch Standardkomponenten bzw. -software der IT realisiert werden, zählen die unteren Schichten (Prozessleit- bis Feldebene) zur Automatisierung, welche die Steuerung und Kontrolle der technischen Anlagen übernimmt. Diese werden auch als Shop-Floor-Ebene bezeichnet. Sie sind durch spezielle Hard- und Softwarelösungen umgesetzt. Die Kommunikation dieser Systeme ist u. a. für spezielle Anwendungsfälle wie harte Echtzeitkommunikation mit Verzögerungen <1ms ausgelegt. Die Integration von Sicherheitsmaßnahmen bei der Kommunikation dieser Systeme stellt oft eine große Herausforderung dar.

\section{Grundprinzipien der sicheren Kommunikation}
\label{Grundlagen:Grundprinzipien der sicheren Kommunikation}
Die Grundprinzipien der sicheren Kommunikation beschreiben die Schutzziele im Bereich der Informationssicherheit. Diese verdeutlichen den Anspruch an die Sicherheit an ein zu implementierendes System oder ein Netzwerk. Sie stellen einen vereinbarten Umfang gegen Bedrohungen dar, welcher von den Kommunikationspartnern gewährleistet wird und nachgewiesen werden kann. Diese klassischen Schutzziele sind auch für Industrie 4.0 Umgebungen zutreffend. Die weitreichende Vernetzung der Systeme in der Industrie 4.0 erfordert jedoch weitere Schutzziele, um einen rechtskonformen Umgang oder besondere Anforderungen sicherzustellen.

\subsection{klassische Schutzziele}
\begin{itemize}
  \item Vertraulichkeit/Zugriffsschutz
  \item (Daten)-Integrität/Änderungsschutz
  \item Authentizität/Fälschungsschutz
  \item Verfügbarkeit
\end{itemize}

\subsection{weitere Schutzziele}
\begin{itemize}
  \item Verbindlichkeit/Nichtabstreitbarkeit
  \item Anonymität
\end{itemize}

TODO - gefällt mir nicht. Muss man die Begriffe erklären?
TODO - ref. \cite{BMWiSec2016}

\subsection{Security by Design}
In der Vergangenheit wurden Sicherheitsmechanismen üblicherweise nachträglich und reaktiv in die Entwicklung von Komponenten mit einbezogen. Industrie 4.0 Umgebungen erfordern umfassende Maßnahmen, um die in \autoref{Grundlagen:Grundprinzipien der sicheren Kommunikation} beschriebenen Schutzziele zu erfüllen und eine sichere Kommunikation zu gewährleisten. Dies gilt vor allem für Maschinenbau- und Fertigungsunternehmen, welche häufig proprietäre Individualsoftware zur Steuerung der Maschinen einsetzen \cite{DTAG2016}. Aus der Notwendigkeit, Sicherheitsaspekte bereits in die Softwareentwicklung mit einzubeziehen und einen Schutz der Kommunikation zu gewährleisten, hat sich der Begriff Security by Design entwickelt.

Die Methoden und Ziele der Angreifer stehen jedoch auch unter einem ständigen Wandel. Somit ist es nicht möglich, eine Securityimplementierung zu entwickeln und diese wiederholt einzusetzen. Vielmehr ist es notwendig, die Sicherheit durch Security by Design so weit als möglich proaktiv herzustellen und gleichzeitig im Schadensfall flexibel und rasch zu reagieren, um das Schadensausmaß zu begrenzen. Es sind Maßnahmen zur Prävention, Detektion und Reaktion erforderlich. TODO - ref. Umsetzungsstrategie Industrie 4.0

\section{Kommunikationsstrukturen in Industrie 4.0 Umgebungen}
Um die Kommunikation zwischen verschiedenen Teilnehmern zu ermöglichen, ergeben sich in der Praxis unterschiedliche Strukturen. Jede dieser Strukturen bietet, je nach Anwendungsfall und zu erfüllenden Anforderungen, Vor- und Nachteile.

TODO - mehr -> siehe sichere Kommunikation-i4.0

\subsection{End2End}
Die Komponenten der Industrie 4.0 Umgebung kommunizieren über einen direkten Kanal miteinander. Dies setzt voraus, dass sich beide Teilnehmer in einem Netzwerk befinden, welches die benötigten Dienste wie z. B. \ac{IP} und \ac{DNS} zur Kommunikation bereitstellt. Des weiteren müssen beide Systeme diese Dienste und Protokolle unterstützen.

\subsection{Gateways}
Um existierende Systeme, welche selbst nicht Industrie 4.0 konform kommunizieren oder zu wenig Rechenleistung besitzen, in die Industrie 4.0 Welt zu integrieren, werden Industrie 4.0 Gateways genutzt. Dabei ist jedoch zu beachten, dass die Systeme hinter den Gateways nicht als Industrie 4.0 Komponenten entwickelt wurden und somit auch keine oder nur wenige dieser Eigenschaften besitzen. Des Weiteren ist es möglich, dass die Kommunikation aus Leistungsgründen oder besonderer Anforderungen über optimierte, proprietäre Protokolle stattfindet. Die Gateways müssen auf die Systeme und deren Protokolle individuell konfiguriert werden, um die Funktionalitäten im Industrie 4.0 Netz bereitstellen zu können, und die Kommunikation zu schützen.

\subsection{Publish-Subscribe}
Das Publish-Subscribe Modell bietet die Möglichkeit Informationen an mehrere Teilnehmer zu verteilen. Hierbei melden sich die Empfänger beim Verteiler an und wählen aus, über welche Nachrichtentypen sie informiert werden möchten. Diese Verteildienste nutzen zur besseren Skalierung und Reduzierung der Netzlast häufig Datagramme wie \ac{UDP}. Durch die Nutzung von Datagrammen geht jedoch die Fehlertoleranz verloren. Somit muss entweder dafür gesorgt werden, dass eine sehr zuverlässige Netzwerkinfrastruktur vorhanden ist und hohe Bandbreitenreserven geschaffen werden, um die Dienstgüte (\ac{QoS}) sicherzustellen oder dieses Modell nur für fehlertolerante Kommunikation wie z. B. Audio- und Video-Anwendungen oder Businessprozesse zu nutzen. 

\subsection{Kommunikation mit Netzwerk als Partner}
Zeitkritische Automatisierungsanwendungen verlangen besondere Netzwerkeigenschaften. Sie können auf Latenz oder Jitter angewiesen sein. Um diese Eigenschaften sicherzustellen, ist es sinnvoll in diese Netze eine Industrie 4.0 Schnittstelle zu integrieren. Somit ist es den Teilnehmern möglich, über die Verwaltungsschale sicherzustellen, dass das Netzwerk die erforderlichen Anforderungen bereitstellt. \cite{sichKom2017}

TODO - Bilder -> sichere-kommunikation-i40

\section{Normen und Standards}
Im Gegensatz zur Industrie 3.0, in welcher Daten auf lokaler Ebene oder zwischen einzelnen internen Unternehmensebenen ausgetauscht wurden, stellt der Datenaustausch und Informationsfluss im vermaschten Industrie 4.0 Netzwerk einen wesentlichen Bestandteil dar. Aktuell gibt es zwei Architekturmodelle zur Umsetzung von Industrie 4.0 Umgebungen. Diese setzen sich aus dem von der Platform Industrie 4.0 entwickelten \ac{RAMI4.0} und der \ac{IIRA} der \ac{IIC} zusammen. Beide Modelle verfolgen verschiedene Integrationsansätze.

Des Weiteren findet die Kommunikation in der Industrie 4.0 nicht mehr über einzelne, vorgegebene Schnittstellen statt, sondern direkt von den Produktionssystemen, also den unteren Ebenen der Automatisierungspyramide. Um dies zu ermöglichen, ist es notwendig, eine einheitliche Kommunikation durch Normen und Standards herzustellen, um eine unternehmensübergreifende Kommunikation dieser Shop-Floor IT zu ermöglichen. 

\subsection{TCP/IP Referenzmodell}
Das TCP/IP Referenzmodell ist ein Schichtenmodell, welches die vier Schichten der Internetprotokollfamilie beschreibt. Sie setzen sich aus Application-, Transport-, Internet- und Link-Layer zusammen. Die Schichten des TCP/IP Referenzmodells überlagern sich mit den Schichten des ISO/OSI Referenzmodells.

\subsubsection{Application Layer}
Die Anwendungsschicht ist für die Übertragung der Nutzdaten zwischen verschiedenen Anwendungen zuständig. Dabei kann es sich um entfernte Anwendungen handeln. Diese sollen sich für den Benutzer verhalten, als würden sie lokal ausgeführt werden.

TODO - Prozess- und Businesslogik

\subsubsection{Transport Layer}
Die Transportschicht sorgt für die Kommunikation zwischen Prozessen. Die Transportschicht nutzt Ports um verschiedene Dienste zu adressieren. Sie beeinflusst, ob es sich um eine zuverlässige Verbindung ( TCP ) oder nicht ( UDP ) handelt.

TODO - End2End Security

\subsubsection{Internet Layer}
Die Internetschicht wird genutzt, um Daten von einem Teilnehmer im Netzwerk zum anderen zu übertragen. Die Endpunkte im Netzwerk werden durch IP Adressen beschrieben.

\subsubsection{Link Layer}
Der Bitübertragungsschicht beschreibt die Topologie des Netzwerks. Sie stellt die physikalische Verbindung der Netzwerkteilnehmer zur Verfügung.

TODO - Bild Interenetprotokollfamilie

\subsection{Industrie 4.0 Referenzarchitekturen}
\subsubsection{\ac{RAMI4.0}}
Um eine flächendeckende Vernetzung zu ermöglichen, muss eine einheitliche Kommunikation geschaffen werden. Die \ac{RAMI4.0} ist eine dreidimensionale Darstellung aller Teilnehmer einer Industrie 4.0 Umgebung und stellt ein Modell einer \ac{SOA} dar. Sie soll eine Verwaltungsschale für Teilnehmer bilden, um eine standardisierte Kommunikation und einfache Inbetriebnahme neuer Komponenten ermöglichen. \cite{rami2016} Die Achsen des \ac{RAMI4.0} bestehen aus:

\begin{itemize}
  \item Achse 1 - Die Hierarchie zeigt die Anlagen, Maschinen sowie das Endprodukt, welche miteinander Vernetzt sind. In diesem Netzwerk werden Funktionen bereitgestellt und Daten ausgetauscht.
  \item Achse 2 - Die Architektur beschreibt - TODO
  \item Achse 3 - Der Produktlebenszyklus wird im Gegensatz zur Industrie 3.0 in das Netzwerk mit eingebunden. Der gesamte Prozess der Produktion, Wartung bis hin zur Verschrottung soll digital erfasst werden.
\end{itemize}

\begin{figure}[h]
  \centering
  \includegraphics[width=15cm]{rami40}
  \caption{RAMI 4.0 - \cite{rami2016}}
  \label{Kap2:RAMI 4.0}
\end{figure}

\clearpage

Nach dem \ac{RAMI4.0} stellt der Communication Layer das Bindeglied zwischen dem Integration Layer, welcher Eigenschaften der physischen Welt für Computersysteme erreichbar macht, und dem Information Layer, welcher die Funktionsbezogenen Daten beinhaltet, dar. \cite{BMWiNeCon2016} 

TODO - Kommunikation beschreiben

Jeder Teilnehmer der Architektur wird als Asset bezeichnet und besitzt seine eigene Verwaltungsschale, welche als Schnittstelle zum Austausch von Informationen dient. Die Verwaltungsschale ist der Übergang zwischen der physischen zur digitalen Welt.

TODO - genauer auf die einzelnen Komponenten eingehen! - Assets, Architektur, Komponenten, Verwaltungsschale
TODO - Architektur wichtig SOA beschreiben -> Angriffsvektoren
TODO - siehe DIN 91345
TODO - Anforderungen an diese Komponenten unterschiedlich

\subsubsection{IIRA}
TODO

\subsection{Protokollstandards}
TODO - Durch die vorausgesetzte M2M-Kommunikation wurde die Entwicklung neuer Protokolle zum effizienten Informationsaustausch vorangetrieben, welche es ermöglichen sollen, eine Standardisierung bereitzustellen und somit eine herstellerübergreifende und plattformunabhängige Kommunikation zu ermöglichen.

\subsubsection{\ac{OPC UA}}
TODO

\subsubsection{DDS}
TODO

\section{Testsystem}
\label{Grundlagen:Testsystem}
Die aus der Analyse hervorgehenden möglichen Schwachstellen im Bereich der Netzwerksicherheit und deren Auswirkungen werden anhand eines vorhandenen, prototypischen Industrie 4.0 Testsystems \cite{Weber2018} veranschaulicht. Das vorhandene System setzt die drei Schichten der Software-Architektur (Verteilungs-, Baustein- und Laufzeitschicht) nach Starke / Hruschka um. Die Netzwekkommunikation wird über das Protokoll \ac{OPC UA} realisiert, welches die Anforderungen der Industrie 4.0 und \ac{RAMI4.0} umsetzt.

\subsection{Architektur}
Das vorhandene System ist, aufgrund der vorgesehenen Einsatzgebiete Lehre, Integrations- und Sicherheitstests, als \ac{VM} umgesetzt worden. Dies ermöglicht es die Testinfrastruktur vom restlichen Netz zu kapseln. Das Betriebssystem der \ac{VM} stellt eine Firewall bereit, welche unerwünschten Netzwerktraffic von oder zu dem System verhindert. Um eine gute Erweiterbarkeit der Testumgebung und Modularisierung der Komponenten zu erreichen, werden die einzelnen Industrie 4.0 Komponenten mit Hilfe der Containerlösung Docker isoliert ausgeführt, verwaltet und deren Netzwerkkommunikation sichergestellt. Durch den zusätzlichen Einsatz des Deploymentsystems Kubernetes wird ein verteiltes Ausführen des Systems ermöglicht und somit eine gute Skalierbarkeit erreicht. 

\subsection{Komponenten}

\subsubsection{Repository}
\subsubsection{Discovery Server}
\subsubsection{\ac{PKI}}
\subsubsection{Identity Provider}
\subsubsection{Verwaltungsinterface}
\subsubsection{Scheduler}