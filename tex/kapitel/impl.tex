\chapter{Implementierung}

TODO - Zusätzlichen Docker bereitstellen der Interface + Tools zur Durchführung bereitstellt!

TODO - um die Komplexität des Systems gering zu halten werden die vorhandenen Komponenten NodeJS, Angular, Docker, ... zur Implementierung der Funktionalitäten im Bereich der Netzwerksicherheit weiterhin genutzt und erweitert.

\section{\ac{DoS}/\ac{DDoS}}
\subsection{Sockstress}
\label{Impl:Sockstress}
TODO - Ansatz: Kann am Testsystem durchgeführt werden. - sockstress implementierung - Diagramm Auslastung
TODO - https://github.com/defuse/sockstress

\section{\ac{MitM}}
\subsection{DHCP Spoofing}
\label{Impl:DHCP Spoofing}
DHCP Spoofing -> Vortäuschen eines DHCP, schneller Leases verteilen als vorhandener DHCP -> MitM
TODO - impl. 2. DHCP Server implementieren
DHCP Snooping -> theor. Verteidigung gegen Spoofing, Switches erlauben nur bestimmten Ports DHCP Traffic zu versenden

\subsection{\ac{ARP} Spoofing}
TODO - verfälschter Ethernetrahmen -> MAC Manipulation - MitM
TODO - Schutz durch statische Tabellen, \ac{IDS}

\section{TODO}

TODO - Zertifikatstests
TODO - Fuzzing
TODO - Logging
TODO - Mustererkennung
TODO - Schwachstellen in implementierten Protokollen
TODO - Anschluss neuer virtueller Geräte über Gateway
TODO - gefundene Schwachstellen bzw. Fehlkonfigurationen sind Grundlage der Implementierung!
TODO - Analyse dient als Basis zur Wahl der Implementierung und Darstellung der Auswirkungen am Testsystem.