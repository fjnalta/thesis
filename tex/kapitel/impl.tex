\chapter{Implementierung}
TODO - um die Komplexität des Systems gering zu halten werden die vorhandenen Komponenten NodeJS, Angular, Docker, ... zur Implementierung der Funktionalitäten im Bereich der Netzwerksicherheit weiterhin genutzt und erweitert.

\section{Interface}
TODO - Zusätzlichen Docker bereitstellen der Interface + Tools zur Durchführung bereitstellt?

\section{\ac{DoS}/\ac{DDoS}}
\subsection{Sockstress}
\label{Impl:Sockstress}
TODO - Ansatz: Kann am Testsystem durchgeführt werden. - sockstress implementierung - Diagramm Auslastung
TODO - https://github.com/defuse/sockstress

\section{\ac{MitM}}
\subsection{\ac{DHCP} Spoofing}
\label{Impl:DHCP Spoofing}
DHCP Spoofing -> Vortäuschen eines DHCP, schneller Leases verteilen als vorhandener DHCP -> MitM
TODO - impl. 2. DHCP Server implementieren
DHCP Snooping -> theor. Verteidigung gegen Spoofing, Switches erlauben nur bestimmten Ports DHCP Traffic zu versenden

\subsection{\ac{ARP} Spoofing}
TODO - verfälschter Ethernetrahmen -> MAC Manipulation - MitM
TODO - Schutz durch statische Tabellen, \ac{IDS}

\subsection{Sniffing}
TODO - Anschluss neuer virtueller Geräte über Gateway
TODO - MQTT, CoAP

\section{Fehlkonfiguration}
\label{Impl:Fehlkonfiguration}
\subsection{OPC UA Secure Mode}

\section{weitere Sicherheitstests}
TODO - Zertifikatstests
TODO - Fuzzing
TODO - Logging
TODO - Mustererkennung