\chapter{Fazit}
\label{Fazit}
Die Ziele der durchgeführten Arbeiten bestanden zum Einen aus der Analyse der Netzwerkkommunikation in Industrie 4.0 Netzen und zum Anderen aus der Erweiterung des Testsystems anhand der Ergebnisse der Analyse. Die Darstellung der Ergebnisse der Analyse durch verschiedene Anwendungsszenarien war mit Hilfe des vorhandenen Softwarestacks im Testsystem nur  begrenzt möglich. Um das Angriffsszenario eines \ac{MitM} Angriffs mit Hilfe eines Rogue \ac{DHCP} Servers darstellen zu können, wurde die Netzwerkkommunikation der in den Ergebnissen dargestellten Komponenten durch die Virtualisierung von zwei Netzwerken umgesetzt. Die Netzwerke beinhalten zwei weitere \ac{VM}s welche die verschiedenen Komponenten des Netzwerks repräsentieren. Diese bestehen aus \ac{DHCP}, \ac{DNS}, \ac{CoAP} Client und Server sowie Maniplations- und Monitoringsystem. Der \ac{DHCP} stellt die Netzwerkkonfiguration bereit und ermöglicht den Betrieb der weiteren Komponenten. Durch die eigene Verwaltung des \ac{DNS} Servers ist es in Zukunft möglich weitere Anwendungsszenarien am Testsystem darzustellen. Dazu gehören u. a. der in \autoref{Analyse:DNS Amplification} beschriebene Angriff der \ac{DNS} Amplification durch die Generierung eigener Zonen mit extrem vielen \ac{RR}, um eine möglichst große \ac{DNS} Response zu provozieren sowie das \ac{DNS} Spoofing und die Analyse der Sicherheitsmechanismen von \ac{DNSSEC}. Die \ac{CoAP} Komponenten werden zur Durchführung und Visualisierung der Kommunikation im Netzwerk genutzt.

Die Angriffsszenarien sowie die Änderungen am Testsystem wurden zuerst konzipiert und anschließend durchgeführt. Die Erweiterung des Testsystems nimmt, mit Ausnahme der Implementierung eines Verschlüsselungsverfahrens während dem Nachrichtenaustausch der Komponenten, keinen Einfluss auf den bisher implementierten Produktionsprozess und dessen Containervirtualisierung. 

Die Skalierbarkeit des Testsystems bleibt hinsichtlich weiterer dargestellter Komponenten und Prozesse durch das Protokoll \ac{OPC UA} sowie der Erweiterung des virtuellen Netzwerks um weitere \ac{VM}s und zur Analyse neuer Technologien erhalten. 

Die Analyse der Kommunikation in Industrie 4.0 Umgebungen ergab, dass die Übertragung der Daten in diesen Umgebungen auf dem Protokoll \ac{IP} basiert. In Verbindung mit dem Protokoll \ac{IP} werden die Transportprotokolle \ac{TCP} und \ac{UDP} genutzt. Auf diesem vorhandenen Netzwerkstack setzen die beispielhaft analysierten Protokolle der Industrie 4.0 wie \ac{OPC UA}, \ac{DDS} und \ac{CoAP} auf. Die Protokolle \ac{OPC UA} und \ac{DDS} erfüllen die Anforderungen der \ac{RAMI4.0} bzw. \ac{IIRA} und wurden nach dem Prinzip \textit{Security by Design} entwickelt. Diese Protokolle sind abstrakt beschrieben, um den verschiedenen Anforderungen der Industrie 4.0 gerecht zu werden und durch den Verzicht auf Features Latenzen zu optimieren oder den Ressourcenverbrauch minimieren zu können. Diese Konfigurationsmöglichkeiten müssen während der Integration der Komponenten beachtet werden und u. U. durch weitere Sicherheitsmaßnahmen wie der Abschottung des Netzes unterstützt werden, um die \ac{IT}-Sicherheit zu gewährleisten.

Die im Rahmen dieser Arbeit gewählten Anwendungsszenarien sollen die mit der geschichteten Kommunikation im Netzwerk einhergehenden Probleme anhand von bekannten Bedrohungsformen der niedrigen Schichten des \ac{TCP}/\ac{IP} Referenzmodells verdeutlichen und die Auswirkungen während der Nutzung von Industrie 4.0 Protokollen auf der Anwendungsschicht darstellen. Das Testsystem ermöglicht es dies mit Hilfe der eingeführten Konfigurationsparameter und des bereitgestellten \ac{GUI}.

Die durchgeführte Analyse der Netzwerkkommunikation beschränkt sich auf beispielhafte Protokolle und deren Dienste als Vertreter der Schichten im \ac{TCP}/\ac{IP} Referenzmodell. In den heterogenen Industrie 4.0 Umgebungen werden weitere Technologien wie \ac{MQTT}, \ac{LDAP} und \ac{TLS} bzw. \ac{DTLS} genutzt oder Authentifizierungsmechanismen wie \ac{OAuth} und ein Zertifikatsmanagement mit Hilfe einer \ac{PKI} bereitgestellt. Diese wurden im Rahmen der Arbeit nicht beschrieben und bieten Ansatzpunkte für nachfolgende Arbeiten. Durch die Implementierung eines Authentifizierungsmechanismus wäre die Möglichkeit gegeben, die Angriffsform des \textit{Bruteforce} sowie weiterer Techniken des Herausfinden von Benutzerdaten in Bezug auf das Protokoll \ac{OPC UA} zu untersuchen. Des Weiteren setzt das genutzte Testsystem die Kommunikation mit \ac{IP}v4 voraus. Dies ist für heutige Industrienetzwerke noch zutreffend, jedoch wird eine stetige Migration zum Protokoll \ac{IP}v6 stattfinden. Eine Migration des Systems zum Protokoll \ac{IP}v6 stellt ebenfalls eine Erweiterung des Testsystems dar.

Der Prozess der 4. industriellen Revolution ist nicht abgeschlossen. Es ist abzuwarten, welche Technologien sich in Zukunft, vor allem im Bereich der \ac{M2M} Kommunikation, in der Industrie etablieren werden. Das Testsystem bietet mit seiner gekapselten Umgebung und der Nutzung von virtuellen Maschinen gute Voraussetzungen für zukünftige Technologien anwendbar und erweiterbar zu sein.