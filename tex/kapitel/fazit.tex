\chapter{Fazit}
Da die grundlegende Netzwerkstruktur der \ac{TCP}/\ac{IP} Netzwerke für Industrie 4.0 Kommunikation übernommen wird, sind auch die damit zusammenhängenden Voraussetzungen und Sicherheitsgedanken zu beachten. \cite{sichKom2017}

Vielzahl von Angriffen auf die verschiedenen Schichten des \ac{TCP}/\ac{IP} Referenzmodells, welches in Industrie 4.0 Umgebungen genutzt wird.
In der Thesis wurden nur wenige Beispielhafte Angriffe dargestellt und durchgeführt.

Analyse der Netzwerkkommunikation in Industrie 4.0 Umgebungen und Erweiterung einer protoypischen Security Testumgebung zur Darstellung verschiedener Berohungsfaktoren

Security Mechanismen sind nicht umsonst und beeinträchtigen die Performance. Security sollte daher nur dort zur Anwendung kommen, wo sie auch benötigt wird. Diese Entscheidung soll aber nicht der Entwickler / Produktmanager treffen, sondern der Anlagenbetreiber (Systemadministration).

Die Nutzung von \ac{OPC UA} bietet keinen automatischen Schutz der IT-Infrastruktur - Konfiguration ist notwendig. Obwohl Securiy by Design. Basiert auf Unteren Schichten.

TODO - Applikationssicherheit != Netzwerksicherheit != Betriebssystemsicherheit
TODO - Schutz auf allen Ebenen -> z.B. OPC UA basiert auf IP-Netz -> Angriffsvektoren von IP und genutzten Diensten immer noch zutreffend