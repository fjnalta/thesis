\chapter{Analyse}

\section{Übertragungsmedien}
\subsection{Kabelgebunden}
\subsection{Funk}
\subsubsection{\ac{GSM}}
\subsubsection{\ac{HSDPA}}
\subsubsection{\ac{LTE}}
\subsubsection{\ac{LPWA}}

\section{Integrationsansätze}
\subsection{Konsolidierung der Netzwerkkommunikation}
TODO - siehe Testsystem Martin - alles spricht OPC UA

\subsection{Gatewaykommunikation}
TODO - siehe Trumpf, axoom -> Gateways übersetzen von heterogener Netzwerkkommunikation in Protokollstandard für unternehmensübergreifende bzw. externe Kommunikation.
Ansatz: Softwareschwachstellen, Softwarefehler, müssen viele Herstellerprotokolle unterstützen - Probleme?

\subsubsection{Security-Komponenten}
\subsubsection{Router}
\subsubsection{Gateways}

\section{Protokolle}
\subsection{HTTP}
\subsection{XMPP}
\subsection{MQTT}
\subsection{CoAP}

\section{Kommunikationsstack}
\subsection{Physical Layer}
\subsection{Data Link Layer}
\subsection{Network Layer}
\subsection{Transport Layer und End2End Security}
\subsection{Prozess- und Businesslogik - Application Layer}

\section{Probleme bei Migration alter Systeme}
\subsection{Inkompatibilität}
\subsection{spezielle bzw. proprietäre Protokolle}
\subsection{besondere Anforderungen der Shop-Floor-Ebene}
\subsection{Industrial Ethernet}

\section{Angriffsvektoren}
\subsection{Verschlüsselung}
\subsection{Paketversand}
\subsection{TODO}

\section{Maßnahmenkatalog}
\subsubsection{Defense in Depth Strategie - TODO (Kuipers,2006)}
TODO - Beschreibung und Einordnung der Defense in Depth Strategie

\begin{figure}[h]
    \centering
    \includegraphics[width=15cm]{defense-in-depth-strategie}
    \caption{Defense in Depth Strategie - TODO ref. Kuipers,2006}
    \label{Kap3:Defense-in-Depth}
\end{figure}