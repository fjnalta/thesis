\chapter{Einleitung}
Mit der heutigen, immer weiter fortschreitenden Vernetzung von Geräten aus Unternehmensinfrastrukturen und Heimnetzen über das Internet, erfährt die Industrie und deren Wertschöpfung einen strukturellen Wandel. Im Gegensatz zur Industrie 3.0, in der die Kommunikation der Geräte nur innerhalb einer Produktionsstätte oder eines Unternehmens stattgefunden hat, erstreckt sich die Kommunikation in Industrie 4.0 Umgebungen über die Unternehmensgrenzen hinweg. Es werden Konzepte zur Einbindung aller Komponenten eines Firmenprozesses, welcher z. B. Produktion, Service- Instandhaltungsaufgaben beinhaltet, realisiert. Diese Systeme kommunizieren miteinander und nutzen dafür immer häufiger eine Ethernet Netzwerkwerkstruktur. Dies setzt die Produktionsanlagen sowie die genutzten Softwaresysteme den gleichen potentiellen Gefahren durch Viren, Würmer oder Trojaner aus, wie reguläre Büro- oder Heim-PC.

Viele \ac{KRITIS}, wie Produktionsanlagen zur Energie- und Wasserversorgung nutzen automatisierte Prozesssteuerungssysteme, \ac{IPC}, \ac{SPS} und \ac{SCADA} Systeme zur Steuerung der Abläufe in den Produktionsanlagen zwischen verteilten Systemen. Die ständige Verfügbarkeit und Überwachung dieser Dienste ist für eine funktionierende Infrastruktur essentiell. Systeme der \ac{KRITIS} können nicht angehalten werden, um Sicherheitsupdates und einen anschließenden Systemneustart durchzuführen. Bei vielen dieser Prozesssteuerungssystemen wurde der Aspekt der IT-Sicherheit nicht berücksichtigt, da eine Vernetzung der Systeme im heutigen Ausmaß nicht vorgesehen war. Die Systeme bieten keine Möglichkeit der Verschlüsselung des Datenverkehrs oder der Authentifizierung der Benutzer.

Die Sicherheit der Produktionsanlagen und deren Netzwerkkommunikation spielt für ein Unternehmen im Industrie 4.0 Umfeld mit Hinblick auf Verfügbarkeit, Zuverlässigkeit und Authentizität eine essentielle Rolle. Sollte es durch Angriffe möglich sein, die Produktion zu sabotieren oder Anlagen und Systeme zu manipulieren, so können die Folgen schwerwiegend sein. Es kann zu Produktionsausfällen kommen und es können Vertragsstrafen drohen.
Ein bekannter Angriff wurde im Jahr 2016 auf das Netz des deutschen Bundestages durchgeführt. Dort wurde ein Zusammenbruch der getroffenen Sicherheitsmaßnahmen erreicht. Es wurden über mehrere Monate unbemerkt sensible Daten entwendet. [TODO - Quelle]

TODO - Kleinere Losgrößen -> von Einzelmaschine zu Fabrik
TODO - mehr -> leitfaden-it-security-i40.pdf - Einleitung
TODO - Stuxnet,Duqu -> auf Produktionsanlagen zugegriffen

Die beschriebenen Probleme bei der Umsetzung einer sicheren Kommunikation im Industrie 4.0 Umfeld sowie die dargestellten, erfolgreich durchgeführten Angriffe auf bestehende Infrastrukturen bieten mir einen Anlass, den aktuellen Stand der IT-Sicherheit beim Datenaustausch in einer heterogenen Industrie 4.0 Umgebung zu analysieren und mögliche Risiken aufzuzeigen.

Um das erwünschte Ergebnis zu erhalten, muss im ersten Schritt eine Literaturanalyse durchgeführt werden. Mit Hilfe dieser werden die Grundlagen zur Analyse der Kommunikation geschaffen. 

Anschließend wird die Sicherheitsanalyse der Netzwerkkommunikation in Industrie 4.0 Umgebungen durchgeführt. Diese beinhaltet die Analyse des Kommunikationsstacks der Netzwerkebene und der verwendeten Protokolle sowie Standards.

Zuletzt werden die Ergebnisse der Analyse durch eine prototypische Implementierung und Erweiterung eines vorhandenen Industrie 4.0 Security Testsystems dargestellt und validiert. 

TODO ref. \cite{Halang2016} und \cite{BMWiSuK2016} und \cite{Schleupner2016} und \cite{Sander2014}

TODO - Um die Komplexität zu reduzieren, wird eine umfassende Modularisierung, eine breite Standardisierung und eine durchgängige Digitalisierung benötigt. Diese Anforderungen sind nicht neu, sie sind auch nicht revolutionär sondern die Folge einer permanenten Weiterentwicklung. Diese Evolution ist ein langjähriger Prozess, der schon lange begonnen hat und es existieren bereits Lösungen für viele der nachfolgend skizzierten Anforderungen, die unter anderem auch die zentralen Grundbausteine für Industrie 4.0 sind. (ref. OPC UA - Wegbereiter der Industrie 4.0)