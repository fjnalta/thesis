\chapter{Einleitung}
Mit der heutigen, immer weiter fortschreitenden Vernetzung von Geräten aus Unternehmensinfrastrukturen und Heimnetzen über das Internet, erfährt die Industrie und deren Wertschöpfung einen strukturellen Wandel. Im Gegensatz zur Industrie 3.0, in der die Kommunikation der Geräte nur innerhalb einer Produktionsstätte oder eines Unternehmens stattgefunden hat, erstreckt sich die Kommunikation in Industrie 4.0 Umgebungen über die Unternehmensgrenzen hinweg. Es werden Konzepte zur Einbindung aller Komponenten eines Firmenprozesses, welcher z. B. Produktion, Service- Instandhaltungsaufgaben beinhaltet, realisiert. Diese Konzepte beschreiben Systeme, welche miteinander über eine Ethernet Netzwerkwerkstruktur mit Hilfe des Protokolls \ac{IP} kommunizieren. Die physikalischen Komponenten sollen in der digitalen Welt repräsentiert werden und Informationen sowie Dienstleistungen in einem standardisierten Format bereitstellen. Die Vernetzung der Systeme und die Nutzung der beschriebenen Netzwerkstruktur setzt die Systeme und Produktionsanlagen der Industrie 4.0 den gleichen potentiellen Gefahren aus, wie reguläre Büro- oder Heim-PCs (\cite{Halang2016}).

Die Sicherheit der Produktionsanlagen und deren Netzwerkkommunikation spielt für ein Unternehmen im Industrie 4.0 Umfeld eine essentielle Rolle. Das Eingreifen in das Produktionsnetzwerk und der Zugriff auf eine einzelne vernetzte Komponente können schwerwiegende Folgen für die Produktion sowie die Integrität und Vertraulichkeit der kommunizierten Nutzdaten haben. Viele Produktionsanlagen zur Energie- und Wasserversorgung nutzen automatisierte Prozesssteuerungssysteme, wie \ac{IPC}, \ac{SPS} und \ac{SCADA}, zur Steuerung der Abläufe zwischen verteilten Systemen. Die ständige Verfügbarkeit und Überwachung dieser Dienste ist für eine funktionierende Infrastruktur essentiell. Die Systeme der Produktionsanlagen können nicht angehalten werden, um Sicherheitsupdates und einen anschließenden Systemneustart durchzuführen. Bei vielen dieser Prozesssteuerungssystemen wurde der Aspekt der IT-Sicherheit nicht berücksichtigt, da eine Vernetzung der Systeme im heutigen Ausmaß nicht vorgesehen war (\cite{Schleupner2016}). Die Systeme bieten aufgrund von Ressourcenknappheit keine Möglichkeit der Verschlüsselung des Datenverkehrs oder der Authentifizierung der Benutzer.

Da eine abrupte Umstellung der Infrastruktur auf Industrie 4.0 konforme Komponenten in der Praxis nicht umsetzbar ist, müssen Wege geschaffen werden um vorhandene Komponenten in die Modelle der Industrie 4.0 Netze zu integrieren. Des Weiteren sind Industrienetze heterogen ausgeprägt. Es werden verschiedene Netzwerkstrukturen und Protokolle genutzt, um den bestehenden Anforderungen gerecht zu werden. Dies erschwert die Integration dieser Komponenten in ein einheitliches Netz. Um trotzdem eine externe Kommunikation der Systeme zu ermöglichen, werden Softwarekomponenten genutzt, welche den Netzwerkverkehr in eine standardisierte Form überführen.

Im Rahmen dieses Entwicklungsprozesses ergibt sich die Frage, welche Bedrohungen auf Industriesysteme durch die Vernetzung einwirken können und ob die in den Referenzmodellen beschriebenen Sicherheitsmaßnahmen sowie die Nutzung des bestehenden Netzwerkstacks einen vollständigen Schutz der Kommunikation im Netzwerk bereitstellen. 

Um diese Frage zu beantworten, wird im ersten Schritt eine Literaturrecherche durchgeführt, um einen Überblick über die in der Industrie genutzten Modelle, Protokolle und Lösungen zu erhalten. Auf Basis der Recherche können die Grundlagen für eine Analyse der digitalen Kommunikation der Komponenten im Industrie 4.0 Netzwerk geschaffen und diese durchgeführt werden. Die Ergebnisse der Analyse sollen im folgenden dazu dienen, relevante Anwendungsszenarien zur Darstellung von Bedrohungen der \ac{IT}-Sicherheit in Industrie 4.0 Netzwerken zu modellieren. Diese Anwendungsszenarien werden anschließend mit Hilfe eines erstellten Konzepts, welches ein vorhandenes Testsystem erweitern soll, umgesetzt und sollen zur Demonstration von Bedrohungsfaktoren in Industrie 4.0 Netzwerken zu Lehr- und Testzwecken genutzt werden können.