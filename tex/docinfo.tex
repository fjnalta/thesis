% -------------------------------------------------------
% Daten für die Arbeit
% Wenn hier alles korrekt eingetragen wurde, wird das Titelblatt
% automatisch generiert. D.h. die Datei titelblatt.tex muss nicht mehr
% angepasst werden.

\newcommand{\hsmasprache}{de} % de oder en für Deutsch oder Englisch
% Für korrekt sortierte Literatureinträge, noch preambel.tex anpassen
% und zwar bei \usepackage[main=ngerman, english]{babel},
% \usepackage[pagebackref=false,german]{hyperref}
% und \usepackage[autostyle=true,german=quotes]{csquotes}

% Titel der Arbeit auf Deutsch
\newcommand{\hsmatitelde}{Sicherheitsanalyse der Netzwerkkommunikation in Industrie 4.0 Umgebungen 
und Erweiterung einer prototypischen Industrie 4.0 Security Testumgebung um 
Funktionalitäten im Bereich der Netzwerksicherheit}

% Titel der Arbeit auf Englisch
\newcommand{\hsmatitelen}{TODO - Title EN}

% Weitere Informationen zur Arbeit
\newcommand{\hsmaort}{Mannheim} % Ort
\newcommand{\hsmaautorvname}{Philipp} % Vorname(n)
\newcommand{\hsmaautornname}{Minges} % Nachname(n)
\newcommand{\hsmadatum}{15.07.2018} % Datum der Abgabe
\newcommand{\hsmajahr}{2018} % Jahr der Abgabe
\newcommand{\hsmafirma}{Hochschule Mannheim, Mannheim} % Firma bei der die Arbeit durchgeführt wurde
\newcommand{\hsmabetreuer}{Prof. Sachar Paulus, Hochschule Mannheim} % Betreuer an der Hochschule
\newcommand{\hsmazweitkorrektor}{TODO - Zweitkorrektor} % Betreuer im Unternehmen oder Zweitkorrektor
\newcommand{\hsmafakultaet}{I} % I für Informatik
\newcommand{\hsmastudiengang}{IB} % IB IMB UIB IM MTB

% Zustimmung zur Veröffentlichung
\setboolean{hsmapublizieren}{true}   % Einer Veröffentlichung wird zugestimmt
\setboolean{hsmasperrvermerk}{false} % Die Arbeit hat keinen Sperrvermerk

% -------------------------------------------------------
% Abstract

% Kurze (maximal halbseitige) Beschreibung, worum es in der Arbeit geht auf Deutsch
\newcommand{\hsmaabstractde}{Nach der Einführung des Begriffs „Industrie 4.0“ im Jahr 2011 und dem gleichzeitigen Start der 4. industriellen Revolution werden Kommunikationsnetze in der Industrie immer mehr zur Automatisierung der Produktion von Gütern oder zum unternehmensinternen sowie -externen Datenaustausch genutzt. Um diese Echtzeitkommunikation oder auch Möglichkeiten der Fernwartung zu gewährleisten, werden immer mehr Anlagen mit Netzwerkzugängen ausgestattet. Die Kommunikation der Industrie 4.0 Netze und Systeme findet Unternehmensübergreifend über einen unsicheren Kanal statt und kann somit ohne bereitgestellte Sicherheitsmaßnahmen genauso angegriffen werden, wie herkömmliche Heim- oder Büronetzwerke. Das Ziel dieser Arbeit ist es zum einen, die Netzwerkkommunikation zwischen Industrie 4.0 Komponenten anhand aktueller Standards zu analysieren, mögliche Angriffsvektoren darzustellen und deren Eintrittswahrscheinlichkeit sowie Schaden zu bewerten. Zum anderen wird ein vorhandenes Industrie 4.0 Security Testsystem anhand der gewonnenen Erkenntnisse im Bereich der Netzwerksicherheit zu Lehr- und Testzwecken erweitert.}

% Kurze (maximal halbseitige) Beschreibung, worum es in der Arbeit geht auf Englisch

\newcommand{\hsmaabstracten}{TODO - Abstract EN}
