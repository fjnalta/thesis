% -------------------------------------------------------
% Daten für die Arbeit
% Wenn hier alles korrekt eingetragen wurde, wird das Titelblatt
% automatisch generiert. D.h. die Datei titelblatt.tex muss nicht mehr
% angepasst werden.

\newcommand{\hsmasprache}{de} % de oder en für Deutsch oder Englisch
% Für korrekt sortierte Literatureinträge, noch preambel.tex anpassen
% und zwar bei \usepackage[main=ngerman, english]{babel},
% \usepackage[pagebackref=false,german]{hyperref}
% und \usepackage[autostyle=true,german=quotes]{csquotes}

% Titel der Arbeit auf Deutsch
\newcommand{\hsmatitelde}{Analyse der Netzwerkkommunikation in Industrie 4.0 Umgebungen und Erweiterung einer protoypischen Security Testumgebung zur Darstellung von Bedrohungsfaktoren}

% Titel der Arbeit auf Englisch
\newcommand{\hsmatitelen}{Analysis of network communication in Industry 4.0 environments and extension of a prototypical security test environment for the presentation of threat factors}

% Weitere Informationen zur Arbeit
\newcommand{\hsmaort}{Mannheim} % Ort
\newcommand{\hsmaautorvname}{Philipp} % Vorname(n)
\newcommand{\hsmaautornname}{Minges} % Nachname(n)
\newcommand{\hsmadatum}{15.07.2018} % Datum der Abgabe
\newcommand{\hsmajahr}{2018} % Jahr der Abgabe
\newcommand{\hsmafirma}{Hochschule Mannheim, Mannheim} % Firma bei der die Arbeit durchgeführt wurde
\newcommand{\hsmabetreuer}{Prof. Dr, Sachar Paulus, Hochschule Mannheim} % Betreuer an der Hochschule
\newcommand{\hsmazweitkorrektor}{Prof. Dr. Maximilian Hauske, Hochschule Mannheim} % Betreuer im Unternehmen oder Zweitkorrektor
\newcommand{\hsmafakultaet}{I} % I für Informatik
\newcommand{\hsmastudiengang}{IB} % IB IMB UIB IM MTB

% Zustimmung zur Veröffentlichung
\setboolean{hsmapublizieren}{true}   % Einer Veröffentlichung wird zugestimmt
\setboolean{hsmasperrvermerk}{false} % Die Arbeit hat keinen Sperrvermerk

% -------------------------------------------------------
% Abstract

% Kurze (maximal halbseitige) Beschreibung, worum es in der Arbeit geht auf Deutsch
\newcommand{\hsmaabstractde}{Nach der Einführung des Begriffs Industrie 4.0 im Jahr 2011 und dem gleichzeitigen Start der 4. industriellen Revolution werden Kommunikationsnetze in der Industrie zur Automatisierung der Produktion von Gütern und dem unternehmensübergreifenden Datenaustausch zur Prozessabwicklung genutzt. In der Folge wurden verschiedene Referenzmodelle und Protokolle zur Standardisierung der Kommunikation in Industrie 4.0 Netzen und der Umsetzung einer sicheren Kommunikation entwickelt. Das Ziel dieser Arbeit ist es, die in der Industrie etablierten Techniken in Bezug auf die Sicherheit der Netzwerkkommunikation zu analysieren und anschließend ein Konzept zu entwickeln, welches es möglich macht anhand von Anwendungsszenarien mögliche Bedrohungen der Vernetzung der Systeme darzustellen.}

% Kurze (maximal halbseitige) Beschreibung, worum es in der Arbeit geht auf Englisch

\newcommand{\hsmaabstracten}{Following the introduction of the term Industry 4.0 in 2011 and the simultaneous launch of the 4th industrial revolution, communication networks in the Industry are used to automate the production of goods and the cross-company data exchange used for process execution. As a result, various reference models and protocols for the standardization of communication in Industrie 4.0 networks and the implementation of secure communication were developed. The aim of this thesis is to analyze the techniques established in the industry regarding the security of the network communication and subsequently to develop a concept which makes it possible to illustrate potential threats of the interconnectedness of the systems based on application scenarios.}
